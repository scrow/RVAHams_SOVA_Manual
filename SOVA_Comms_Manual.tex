%%%%%%%%%%%%%%%%%%%%%%%%%%%%%%%%%%%%%%%%%%%%%%%%%%%%%%%%%%%%%%%%%%%%%%%%
%
% This work is licensed under the Creative Commons
% Attribution-NonCommercial-ShareAlike 4.0 International License. To
% view a copy of this license, visit
% http://creativecommons.org/licenses/by-nc-sa/4.0/.
%
% Portions of this work such as stylesheets and resource files may be
% subject to other licenses or terms.  Refer to those files or related
% documentation for additional information.
%
%%%%%%%%%%%%%%%%%%%%%%%%%%%%%%%%%%%%%%%%%%%%%%%%%%%%%%%%%%%%%%%%%%%%%%%%

\documentclass[pdflatex,letterpaper,twoside,12pt]{book}

\title             {SOVA Comms Manual}
\author            {Maintained by:\\Matt Kimball K4MTK\\Steve Crow KG4PEQ}
\date              {Effective 01-Aug-2015}
\newcommand\docver {Version 2016.1}

\def \logofile     {sty/RVAlogo_v3_72dpi.jpg}

\usepackage{sty/rvahamsbook}
\rvahFormat{printing}
\disableAutoNumbering
\useColorLinks

\printWatermark{DRAFT}

\begin{document}
\rvahTitlePage
\skipToTOC
\rvahTOC

%%%%%%%%%%%%%%%%%%%%%%%%%%%%%%%%%%%%%%%%%%%%%%%%%%%%%%%%%%%%%%%%%%%%%%%%
%%%%%%%%%%%%%%%%%%%%%%%%%%%%%%%%%%%%%%%%%%%%%%%%%%%%%%%%%%%%%%%%%%%%%%%%
%%%%%%%%%%%%%%%%%%%%%%%%%%%%%%%%%%%%%%%%%%%%%%%%%%%%%%%%%%%%%%%%%%%%%%%%
%%%%%%%%%%%%%%%%%%%%%%%%%%%%%%%%%%%%%%%%%%%%%%%%%%%%%%%%%%%%%%%%%%%%%%%%
%%%%%%%%%%%%%%%%%%%%%%%%%%%%%%%%%%%%%%%%%%%%%%%%%%%%%%%%%%%%%%%%%%%%%%%%
%%%%%%%%%%%%%%%%%%%%%%%%%%%%%%%%%%%%%%%%%%%%%%%%%%%%%%%%%%%%%%%%%%%%%%%%

\chapter{About}

\section{License}

This work is licensed under the Creative Commons Attribution-NonCommercial-ShareAlike 4.0 International License. To view a copy of this license, visit\\
\href{http://creativecommons.org/licenses/by-nc-sa/4.0/}{http://creativecommons.org/licenses/by-nc-sa/4.0/}.

\section{Project Source Code and Adaptations}

The source code for this manual is available for adaptation from the following Github repository:

\href{https://github.com/scrow/rvahams-doc-sova-manual}{https://github.com/scrow/rvahams-doc-sova-manual}

You are welcome to adapt this manual to your needs provided attribution and subsequent licensing is compliant with the terms of the Creative Commons Attribution-NonCommercial-ShareAlike 4.0 International License.

%%%%%%%%%%%%%%%%%%%%%%%%%%%%%%%%%%%%%%%%%%%%%%%%%%%%%%%%%%%%%%%%%%%%%%%%
%%%%%%%%%%%%%%%%%%%%%%%%%%%%%%%%%%%%%%%%%%%%%%%%%%%%%%%%%%%%%%%%%%%%%%%%
%%%%%%%%%%%%%%%%%%%%%%%%%%%%%%%%%%%%%%%%%%%%%%%%%%%%%%%%%%%%%%%%%%%%%%%%
%%%%%%%%%%%%%%%%%%%%%%%%%%%%%%%%%%%%%%%%%%%%%%%%%%%%%%%%%%%%%%%%%%%%%%%%
%%%%%%%%%%%%%%%%%%%%%%%%%%%%%%%%%%%%%%%%%%%%%%%%%%%%%%%%%%%%%%%%%%%%%%%%
%%%%%%%%%%%%%%%%%%%%%%%%%%%%%%%%%%%%%%%%%%%%%%%%%%%%%%%%%%%%%%%%%%%%%%%%

\chapter{Introduction}

\section{Welcome}

Welcome to the Special Olympics Virginia Communications Team!

The SOVA Comms Team provides a unique package of services to Special Olympics Summer Games, combining our understanding of their needs with our equipment, technical knowledge, and professional standards to provide a \emph{total communications solution}.

Whether this is your first time participating as a communications volunteer for Special Olympics or you have joined us many times before, please take the time to review this manual each year.  It contains all the information you should need to better understand this event along with your roles and responsibilities.

%%%%%%%%%%%%%%%%%%%%%%%%%%%%%%%%%%%%%%%%%%%%%%%%%%%%%%%%%%%%%%%%%%%%%%%%
%%%%%%%%%%%%%%%%%%%%%%%%%%%%%%%%%%%%%%%%%%%%%%%%%%%%%%%%%%%%%%%%%%%%%%%%
%%%%%%%%%%%%%%%%%%%%%%%%%%%%%%%%%%%%%%%%%%%%%%%%%%%%%%%%%%%%%%%%%%%%%%%%
%%%%%%%%%%%%%%%%%%%%%%%%%%%%%%%%%%%%%%%%%%%%%%%%%%%%%%%%%%%%%%%%%%%%%%%%
%%%%%%%%%%%%%%%%%%%%%%%%%%%%%%%%%%%%%%%%%%%%%%%%%%%%%%%%%%%%%%%%%%%%%%%%
%%%%%%%%%%%%%%%%%%%%%%%%%%%%%%%%%%%%%%%%%%%%%%%%%%%%%%%%%%%%%%%%%%%%%%%%

\chapter{General Event Information}

\section{Nature of the Event}

The Special Olympics Summer Games is a two-day event located at the University of Richmond and four offsite locations around town.  Activities include track and field, weight lifting, swimming, softball, and bowling.  The event also features Healthy Athletes, which provides eye exams, dental exams, and general health checkups.

Athletes and volunteers begin to arrive on-site Thursday afternoon and most stay at University of Richmond until Sunday morning.  Athletic events take place Friday and Saturday, with opening ceremonies in the Robins Center Friday evening and a dance Saturday night.

%%%%%%%%%%%%%%%%%%%%%%%%%%%%%%%%%%%%%%%%%%%%%%%%%%%%%%%%%%%%%%%%%%%%%%%%
%%%%%%%%%%%%%%%%%%%%%%%%%%%%%%%%%%%%%%%%%%%%%%%%%%%%%%%%%%%%%%%%%%%%%%%%

\section{Our Role in the Event}

Amateur Radio operators help our community through direction and professional communication between volunteer coordinators and Special Olympic management staff.  Communications volunteers and Amateur Radio operators are also responsible for assisting and directing spectators and athletes alike.

%%%%%%%%%%%%%%%%%%%%%%%%%%%%%%%%%%%%%%%%%%%%%%%%%%%%%%%%%%%%%%%%%%%%%%%%
%%%%%%%%%%%%%%%%%%%%%%%%%%%%%%%%%%%%%%%%%%%%%%%%%%%%%%%%%%%%%%%%%%%%%%%%

% \section{History}

%%%%%%%%%%%%%%%%%%%%%%%%%%%%%%%%%%%%%%%%%%%%%%%%%%%%%%%%%%%%%%%%%%%%%%%%
%%%%%%%%%%%%%%%%%%%%%%%%%%%%%%%%%%%%%%%%%%%%%%%%%%%%%%%%%%%%%%%%%%%%%%%%
%%%%%%%%%%%%%%%%%%%%%%%%%%%%%%%%%%%%%%%%%%%%%%%%%%%%%%%%%%%%%%%%%%%%%%%%
%%%%%%%%%%%%%%%%%%%%%%%%%%%%%%%%%%%%%%%%%%%%%%%%%%%%%%%%%%%%%%%%%%%%%%%%
%%%%%%%%%%%%%%%%%%%%%%%%%%%%%%%%%%%%%%%%%%%%%%%%%%%%%%%%%%%%%%%%%%%%%%%%
%%%%%%%%%%%%%%%%%%%%%%%%%%%%%%%%%%%%%%%%%%%%%%%%%%%%%%%%%%%%%%%%%%%%%%%%

\chapter{Communications Roles}

%%%%%%%%%%%%%%%%%%%%%%%%%%%%%%%%%%%%%%%%%%%%%%%%%%%%%%%%%%%%%%%%%%%%%%%%
%%%%%%%%%%%%%%%%%%%%%%%%%%%%%%%%%%%%%%%%%%%%%%%%%%%%%%%%%%%%%%%%%%%%%%%%

\section{Communications Director}

The Communications Director is responsible for coordinating the amateur radio team's participation in Special Olympics activities.  The Communications Director meets with key event officials throughout the year to plan for amateur radio involvement and is responsible for the recruitment of communications volunteers and acquisition of all necessary equipment in advance of each year's event.

%%%%%%%%%%%%%%%%%%%%%%%%%%%%%%%%%%%%%%%%%%%%%%%%%%%%%%%%%%%%%%%%%%%%%%%%
%%%%%%%%%%%%%%%%%%%%%%%%%%%%%%%%%%%%%%%%%%%%%%%%%%%%%%%%%%%%%%%%%%%%%%%%

\section{Net Controls}

The primary responsibility of Net Control is to ensure the orderly flow of communications between all stations through the use of a directed net.  Net Control maintains a list of communications volunteers, their assigned locations, and current status.  Requests for event status, weather, other information, and supplies are routed through either of two Net Controls, depending on the type of traffic.

\subsection{Primary Net Control}

The Primary Net Control station is manned by two volunteers who handle all traffic except for requests for physical assets which are fulfilled by Logistics Net Control.

Primary Net Control marks all stations in and out of service, keeps a record of the status of events (on time, ahead of schedule, running behind), and keeps key event staff up to date on event status, weather, and other critical details at prescribed intervals.

\subsection{Logistics Net Control}

Logistics Net Control handles all requests for supplies.  Any request for water, ice, tables, chairs, missing awards, microphones, or any other physical piece of equipment would be sent to Logistics Net Control, who has radio contact with University of Richmond staff, Special Olympics security personnel, and the medical team via other radio systems.  Logistics Net Control takes the request, relays it over other channels, and optionally provides and update to the requesting party when the resource has been fulfilled.

\subsection{Backup Net Control}

Backup Net Control monitors the designated backup repeater frequency for any calls, allowing for a quick transition to the backup repeater should congestion, interference, or other downtime impede communications on the main repeater.

Backup Net Control also fulfills requests for information on coaches and athletes, making and receiving phone calls as necessary to locate or reunite lost participants.

%%%%%%%%%%%%%%%%%%%%%%%%%%%%%%%%%%%%%%%%%%%%%%%%%%%%%%%%%%%%%%%%%%%%%%%%
%%%%%%%%%%%%%%%%%%%%%%%%%%%%%%%%%%%%%%%%%%%%%%%%%%%%%%%%%%%%%%%%%%%%%%%%

\section{Shadow Units}

Shadows are assigned to key event staff and will follow them from point to point, keeping within arm's reach at all times.  The role of the Shadow is to send and receive traffic to and from their assigned event staff member.  This traffic may consist of requests for information on event status, supplies, or location.  Shadows may also need to take and hold onto messages for their assigned staff member during meetings and other activities where immediate communication is not possible.

Shadows are highly mobile, almost constantly on the move, and will likely walk over several miles each day.

%%%%%%%%%%%%%%%%%%%%%%%%%%%%%%%%%%%%%%%%%%%%%%%%%%%%%%%%%%%%%%%%%%%%%%%%
%%%%%%%%%%%%%%%%%%%%%%%%%%%%%%%%%%%%%%%%%%%%%%%%%%%%%%%%%%%%%%%%%%%%%%%%

\section{Stationary Units}

Stationary units are assigned to specific locations such as tennis, bocce, swimming, softball, volunteer check-in, delgation check-in, transportation, and the awards tent.  They are responsible for sending and receiving traffic to and from the volunteers at the assigned location.  This traffic may consist of requests for information on event timing or weather, logistics, and lost athletes.

Stationary volunteers don't usually move around more than their local event area.

%%%%%%%%%%%%%%%%%%%%%%%%%%%%%%%%%%%%%%%%%%%%%%%%%%%%%%%%%%%%%%%%%%%%%%%%
%%%%%%%%%%%%%%%%%%%%%%%%%%%%%%%%%%%%%%%%%%%%%%%%%%%%%%%%%%%%%%%%%%%%%%%%

\section{Floaters}

Floaters are ``spare'' communications volunteers who may rotate from one location to another, providing temporary relief during breaks and filling in for volunteers who cannot be at their assigned location.  At the conclusion of an assigned shift, all communications volunteers are welcome (and encouraged) to remain on-site as a floater.  Additional help is always needed and appreciated.

%%%%%%%%%%%%%%%%%%%%%%%%%%%%%%%%%%%%%%%%%%%%%%%%%%%%%%%%%%%%%%%%%%%%%%%%
%%%%%%%%%%%%%%%%%%%%%%%%%%%%%%%%%%%%%%%%%%%%%%%%%%%%%%%%%%%%%%%%%%%%%%%%
%%%%%%%%%%%%%%%%%%%%%%%%%%%%%%%%%%%%%%%%%%%%%%%%%%%%%%%%%%%%%%%%%%%%%%%%
%%%%%%%%%%%%%%%%%%%%%%%%%%%%%%%%%%%%%%%%%%%%%%%%%%%%%%%%%%%%%%%%%%%%%%%%
%%%%%%%%%%%%%%%%%%%%%%%%%%%%%%%%%%%%%%%%%%%%%%%%%%%%%%%%%%%%%%%%%%%%%%%%
%%%%%%%%%%%%%%%%%%%%%%%%%%%%%%%%%%%%%%%%%%%%%%%%%%%%%%%%%%%%%%%%%%%%%%%%

\chapter{Special Olympics Staff Roles}

\begin{description}
  \item[Event Coordinator] Leisha Santili is responsible for food, housing, opening ceremonies, and more.  Most general questions should be directed to her.
  \item[Competition Director] FAKE BROOKE is responsible for all competion events, rules, and issues coming from an event.
  \item[Transportaion Director] Lynn is responsible for coordinating transportation between the external venues and UofR.
  \item[Volunteer Coordinator] Ellen Cosplow is responsible for handling volunteer activities including sending people to venues when requested.
  \item[Security Coordinator] Vicki Keaton is responsible for the security activities and is run on a separate radio system.
  \item[Medical Director] Dennis Spurriur is responsible for all medical personnel and is mainly on a separate radio system.
\end{description}

%%%%%%%%%%%%%%%%%%%%%%%%%%%%%%%%%%%%%%%%%%%%%%%%%%%%%%%%%%%%%%%%%%%%%%%%
%%%%%%%%%%%%%%%%%%%%%%%%%%%%%%%%%%%%%%%%%%%%%%%%%%%%%%%%%%%%%%%%%%%%%%%%
%%%%%%%%%%%%%%%%%%%%%%%%%%%%%%%%%%%%%%%%%%%%%%%%%%%%%%%%%%%%%%%%%%%%%%%%
%%%%%%%%%%%%%%%%%%%%%%%%%%%%%%%%%%%%%%%%%%%%%%%%%%%%%%%%%%%%%%%%%%%%%%%%
%%%%%%%%%%%%%%%%%%%%%%%%%%%%%%%%%%%%%%%%%%%%%%%%%%%%%%%%%%%%%%%%%%%%%%%%
%%%%%%%%%%%%%%%%%%%%%%%%%%%%%%%%%%%%%%%%%%%%%%%%%%%%%%%%%%%%%%%%%%%%%%%%

\chapter{Comms Volunteer Guidelines}

You are an event guide and a Professional Communicator; be visible and answer questions, help out, go above and beyond.  As a volunteer, you will be assisting both visitors and athletes with the Get On The Air station.

\section{Licensing and Equipment Requirements}

Volunteers do not need to be a licensed amateur to participate on the communications team!  However, operating a radio on amateur frequencies will require a valid license of the appropriate class.

Communications volunteers are required to furnish their own radio equipment and should arrive at the event with their radio already programmed per the communications plan, which will be provided to all volunteers in advance of the event.  Limited assistance with programming radios, verifying functionality, and resolving technical issues will be available the day of the event.

When programming a radio for this event, consider all amateur frequencies which will be in use, including on-site and off-site repeaters, simplex frequencies, and backup repeaters.  Take a moment to verify proper tone squelch and repeater offset values.

All volunteers are encouraged to come prepared with an additional battery pack, a headset, and a high-visibility safety vest to aid in visual identification.

%%%%%%%%%%%%%%%%%%%%%%%%%%%%%%%%%%%%%%%%%%%%%%%%%%%%%%%%%%%%%%%%%%%%%%%%
%%%%%%%%%%%%%%%%%%%%%%%%%%%%%%%%%%%%%%%%%%%%%%%%%%%%%%%%%%%%%%%%%%%%%%%%

\section{Volunteer Registration}

First and foremost, \textbf{walk-up volunteers are always welcome!}.  If you decide to join us the day of the event, please check in on-site at the Net Control office to be added to our volunteer roster.

To aid in planning, we need as many volunteers to pre-register for this event as possible.  Registration takes place through the RVAHams web site.  We will collect contact information including e-mail address and cell phone number and will use this information to communicate details about this event.  We will also collect radio equipment and assignment preferences (stationary vs. active position, indoors vs. outdoors, etc).

If plans change after registering and a communications volunteer can no longer work their assigned shift, or if they will be available to work additional hours, the coordinators should be notified as early as possible so we can incorporate these changes into the overall staffin plan.

%%%%%%%%%%%%%%%%%%%%%%%%%%%%%%%%%%%%%%%%%%%%%%%%%%%%%%%%%%%%%%%%%%%%%%%%
%%%%%%%%%%%%%%%%%%%%%%%%%%%%%%%%%%%%%%%%%%%%%%%%%%%%%%%%%%%%%%%%%%%%%%%%

\section{Scheduling}

We will make every effort to accommodate your preference of location, time, and any special needs.  Pre-registering for this event will help us plan accordingly.  Please understand that from time to time we have last-minute needs to fill gaps in scheduling or add volunteers at additional locations, and we appreciate any flexibility you may have in adapting to last-minute changes in assignment and availability.

%%%%%%%%%%%%%%%%%%%%%%%%%%%%%%%%%%%%%%%%%%%%%%%%%%%%%%%%%%%%%%%%%%%%%%%%
%%%%%%%%%%%%%%%%%%%%%%%%%%%%%%%%%%%%%%%%%%%%%%%%%%%%%%%%%%%%%%%%%%%%%%%%

\section{What to Bring With You}

\textbf{Dress appropriately.}  Wear comfortable shoes and light colored clothing.  Bring proper sun protection.  Shadows should consider a lightweight backpack to store bottled water or a hydration pack along with a few small snacks.

\textbf{Radio and accessories.}  An extra battery pack (or two) is highly recommended, especially if scheduled more than a few hours.  This event generates a tremendous amount of radio traffic.  A headset is highly recommended;  a handheld speaker microphone is a good alternative to a headset.  All event locations are very noisy.  It will be extremely difficult to hear the speaker built into a radio.  A headset is \textbf{essential} for any volunteers stationed at swimming or bowling events.

\textbf{Medication, snacks, etc.}  Please remember to pack any prescription or over-the-counter medications.  It will be very difficult for us to fill scheduling gaps caused by volunteers who need to leave temporarily to retrieve forgotten items.

%%%%%%%%%%%%%%%%%%%%%%%%%%%%%%%%%%%%%%%%%%%%%%%%%%%%%%%%%%%%%%%%%%%%%%%%
%%%%%%%%%%%%%%%%%%%%%%%%%%%%%%%%%%%%%%%%%%%%%%%%%%%%%%%%%%%%%%%%%%%%%%%%

\section{Volunteer Check-In}

All volunteers must check-in.  Communications volunteers at any of University of Richmond locations must check in in person.  Operators in the field can check in via net control when arriving on location.  During check-in, communications volunteers should receive a badge, hat, and any other information regarding the assigned post.  To allow adequate time for the check-in process, volunteers should arrive an hour prior to their assignment.  Once on post, communications volunteers should contact Net Control to confirm their position.  Check-in is located on the third floor on the South Concourse of the E. Claibourne Robins Stadium.  The Backup Net Control Operator will assist during the check-in process.

%%%%%%%%%%%%%%%%%%%%%%%%%%%%%%%%%%%%%%%%%%%%%%%%%%%%%%%%%%%%%%%%%%%%%%%%
%%%%%%%%%%%%%%%%%%%%%%%%%%%%%%%%%%%%%%%%%%%%%%%%%%%%%%%%%%%%%%%%%%%%%%%%

\section{Equipment and Comms Checks}

It is important that the Amateur Radio operator comes prepared with any needed equipment.  In the event that the volunteer has an equipment failure, the RVAHams team will make every attempt to keep a backup of equipment and equipment related programing files in the event the communications operator needs assistance with a loaner radio or for programming as may be needed.  Equipment loaned out will be signed out and is expected to be returned in like condition.

Upon checking in at Net Control, there should be ample time to do a communications check of your equipment on the secondary net control frequency.  When initiating a communications check, the Amateur Operator should say “Backup Net Control” and then their call sign followed by the prowords “Radio Check”.

%%%%%%%%%%%%%%%%%%%%%%%%%%%%%%%%%%%%%%%%%%%%%%%%%%%%%%%%%%%%%%%%%%%%%%%%
%%%%%%%%%%%%%%%%%%%%%%%%%%%%%%%%%%%%%%%%%%%%%%%%%%%%%%%%%%%%%%%%%%%%%%%%

\section{Directed Net Basics}

All communications on amateur frequencies will take place in the form of a directed net.  A directed net means that all transmissions will take place with the coordination and direction of Net Control.  Here are a few guidelines when operating in a directed net:

\begin{itemize}

	\item To contact Net Control, stations should key up and wait a few seconds and then call ``Net Control'' followed by their tactical call sign and then \textbf{wait to be recognized}.
	\item If Net Control is busy handling other traffic, every effort will be made to immediately acknowledge calling stations but stations may still be asked to stand by before continuing with the contact.
	\item To contact another station, first make contact with Net Control.  When Net Control responds, stations should then ask to ``go direct'' with the other station.  Net Control may then grant permission for the communication to continue.
	\item \textbf{Before making any call, listen for a few moments to determine whether an existing conversation is in progress.}
	\item Stations feeding to interject information relating to an existing conversation may break in by simply giving their tactical call sign followed by the keyword ``information'' or ``reference'', and may proceed to give that information once acknowledged by either Net Control or any of the stations involved in the conversation.
	\item \textbf{If there is a medical emergency or any other emergency or priority traffic, announce the word ``break'' between transmissions and the net will immediately yield to the emergency traffic.}  The word ``break'' should \textbf{only} be used in emergency situations.

\end{itemize}

%%%%%%%%%%%%%%%%%%%%%%%%%%%%%%%%%%%%%%%%%%%%%%%%%%%%%%%%%%%%%%%%%%%%%%%%
%%%%%%%%%%%%%%%%%%%%%%%%%%%%%%%%%%%%%%%%%%%%%%%%%%%%%%%%%%%%%%%%%%%%%%%%

\section{Tactical Call Signs}

Tactical call signs can prevent confusion, saving a great amount of time and also aid in making a net run smoothly and efficiently.  Tactical call signs are used to identify a specific location or function.  They are especially helpful when multiple operators are being rotated (in shifts) or with the number of volunteers we are managing during this event.  When using tactical call signs, amateur operators only need to identify with their FCC-assigned call sign at the end of a transmission when they do not expect to transmit again within ten minutes and at the end of their shift assignment. (FCC Part 97.119)

\rvahbox{Tactical Call Signs}{
Stations will primarily use tactical call signs during this event.  The use of tactical call signs does not relieve stations of their obligation to identify once every ten minutes or at the conclusion of a conversation, per FCC regulations.}

%%%%%%%%%%%%%%%%%%%%%%%%%%%%%%%%%%%%%%%%%%%%%%%%%%%%%%%%%%%%%%%%%%%%%%%%
%%%%%%%%%%%%%%%%%%%%%%%%%%%%%%%%%%%%%%%%%%%%%%%%%%%%%%%%%%%%%%%%%%%%%%%%

\section{Backup Plans}

In the case of repeater failure, a simplex frequency will be used around the UofR campus.  Cells phones will have to be used for those on external venues.

%%%%%%%%%%%%%%%%%%%%%%%%%%%%%%%%%%%%%%%%%%%%%%%%%%%%%%%%%%%%%%%%%%%%%%%%
%%%%%%%%%%%%%%%%%%%%%%%%%%%%%%%%%%%%%%%%%%%%%%%%%%%%%%%%%%%%%%%%%%%%%%%%

\section{Periodic Status Reports}

As amateur radio operators we are here to add value to the Virginia Special Olympic Summer Games.  We require our operators to call in to net control each hour with situation reports to include the status of the game the communications operator is assigned to: is the event ahead of schedule, on schedule or running behind and if so how far ahead or behind?  If an event is ending early, we need to know before and approximately at what time the event may end.  This information is vital to ensure we can coordinate transportation, adjust awards ceremonies, and maintain a controlled event.

%%%%%%%%%%%%%%%%%%%%%%%%%%%%%%%%%%%%%%%%%%%%%%%%%%%%%%%%%%%%%%%%%%%%%%%%
%%%%%%%%%%%%%%%%%%%%%%%%%%%%%%%%%%%%%%%%%%%%%%%%%%%%%%%%%%%%%%%%%%%%%%%%

\section{Communicating with Net Control}

\subsection{Classifying and Routing Requests}

Radio traffic during any special event falls primarily into two main categories:

\begin{itemize}
	\item \textbf{Informational.}  This type of traffic involves \textbf{non-tangible items} event status, weather, questions about event schedules, location of athletes and coaches, etc.
	\item \textbf{Logistics.}  This type of traffic involves \textbf{tangible items} such as water, ice, medals, microphones, tables, chairs, and other physical assets.
\end{itemize}

All logistics traffic is directed to Logistics Net Control.  All other traffic goes to Primary Net Control.  Calls for informational traffic are best directed to ``Primary Net Control'' or simply ``Net Control''.  Logistics Net Control has contact with individuals capable of fulfilling requests for physical resources.

\rvahbox{Medical Calls}{All calls for medical emergencies and requests for the medical team will be handled by Logistics Net Control.}

\subsection{Handling Resource Requests}

The Amateur Radio operator may at times be requested to obtain certain resources such as ice, water or tables.  Before making a call to Logistics Net Control; it is imperative that the Operator looks to verify the need themself.

The Amateur Radio operator can call into Logistics Net Control, once it is verified that there is a need.  Relay said request to include: the individual making the request, the quantity of the items in the request, the location of the needed items and a time of when the items are needed by.

Logistics control may either ask further questions, inquire further to clarify the needs or will make contact with the appropriate team, and provide the amateur radio operator with an ETA, if/{}as needed.

%%%%%%%%%%%%%%%%%%%%%%%%%%%%%%%%%%%%%%%%%%%%%%%%%%%%%%%%%%%%%%%%%%%%%%%%
%%%%%%%%%%%%%%%%%%%%%%%%%%%%%%%%%%%%%%%%%%%%%%%%%%%%%%%%%%%%%%%%%%%%%%%%

\section{Communicating with Other Units}

The Virginia Special Olympic Summer Games is under the control of a directed net. At no time should any Amateur Radio operator go direct with another without first getting the permission of Net Control. When Net Control grants that permission, the operator should then call the station they need to contact directly. The receiving station should listen and wait for that call to be made before acknowledging, even if they copied the operator asking permission.

%%%%%%%%%%%%%%%%%%%%%%%%%%%%%%%%%%%%%%%%%%%%%%%%%%%%%%%%%%%%%%%%%%%%%%%%
%%%%%%%%%%%%%%%%%%%%%%%%%%%%%%%%%%%%%%%%%%%%%%%%%%%%%%%%%%%%%%%%%%%%%%%%

\section{Handling Medical Issues}

All medical issues must be immediately reported to Logistics Net Control.  In the event of a medical emergency, announce ``break'' and Net Control will immediately yield to the emergency traffic.  When calling Logistics Net Control with any medical issue, indicate whether a member of the medical team is already on scene or whether the medical team needs to be dispatched to the location.

\rvahbox{Do Not Dial 911!}{\textbf{At no time should anyone other than a Net Control Operator dial 911.}  All 911 calls are handled by Net Control after consultation with the on-site Medical Director.}

Be prepared to quickly provide:

\begin{itemize}
	\item The nature of the medical issue.
	\item The exact location where medical assistance is required.
	\item The bib number of the athlete(s) in need of medical assistance.
\end{itemize}

With this information, Logistics Net Control will be able to promptly dispatch medical aid and notify the coach(es).

%%%%%%%%%%%%%%%%%%%%%%%%%%%%%%%%%%%%%%%%%%%%%%%%%%%%%%%%%%%%%%%%%%%%%%%%
%%%%%%%%%%%%%%%%%%%%%%%%%%%%%%%%%%%%%%%%%%%%%%%%%%%%%%%%%%%%%%%%%%%%%%%%

\section{Handling Weather Issues}

Net Control will monitor local weather observations, on-site temperature readings, and weather radar, and will remain in contact with event officials and the National Weather Service in the event inclement weather is expected.  All decisions to delay or cancel events based on weather conditions (such as thundrstorms or excessive heat) will be made by Special Olympics event staff and communicated to the affected locations.

Communications volunteers who are approached by event volunteers, athletes, coaches, or spectators with concerns regarding weather conditions should contact Net Control for assistance.

%%%%%%%%%%%%%%%%%%%%%%%%%%%%%%%%%%%%%%%%%%%%%%%%%%%%%%%%%%%%%%%%%%%%%%%%
%%%%%%%%%%%%%%%%%%%%%%%%%%%%%%%%%%%%%%%%%%%%%%%%%%%%%%%%%%%%%%%%%%%%%%%%

\section{Lost Athletes and Lost Coaches}

Every year several athletes become separated from their group.

When approached by a lost athlete, Communications volunteers should attempt to calm the athlete if necessary, and try to ascertain where the athlete is supposed to be.  All athletes should be marked with a bib number and Net Control has a roster of all participants which includes identification and contact numbers for the coaches.

Net Control will make a call to the coach(es) responsible for the athlete and will provide a status update once contact has been made.  Please be prepared to provide the exact location of the lost athlete.

Volunteers must remain with the lost athlete until he or she can be reuinted with their party.

%%%%%%%%%%%%%%%%%%%%%%%%%%%%%%%%%%%%%%%%%%%%%%%%%%%%%%%%%%%%%%%%%%%%%%%%
%%%%%%%%%%%%%%%%%%%%%%%%%%%%%%%%%%%%%%%%%%%%%%%%%%%%%%%%%%%%%%%%%%%%%%%%

\section{Breaks and Meal Relief}

Periodically, an operator may need a short break; before leaving your post, call into Net Control for permission and advise of your need.  Meals are provided for the communications team at lunch dinner both Friday and Saturday.  Operators anywhere at the event, should work with their Volunteer Coordinators to ensure lunches are provided for them for Saturday Lunch; Mobile and Net Control Operators will either be able to grab lunch on the go with their staff member or it will be retrieved on their behalf by a runner.  Friday and Saturday Evening Dinner can be picked up at the University of Richmond dining hall or gathered in a group and delivered to the South Concourse Net Control Operations area on the third floor.

Once communications volunteers are free and have been released from their post, they should notify Net Control of their meal intention so that the correct number of meals can be picked up or the communications volunteer can go directly to the dinner hall once released from your post.

%%%%%%%%%%%%%%%%%%%%%%%%%%%%%%%%%%%%%%%%%%%%%%%%%%%%%%%%%%%%%%%%%%%%%%%%
%%%%%%%%%%%%%%%%%%%%%%%%%%%%%%%%%%%%%%%%%%%%%%%%%%%%%%%%%%%%%%%%%%%%%%%%

\section{Volunteer Check-Out}

Communications volunteers are required to call into Net Control before checking out and leaving their post. Net Control may need additional assistance in filling a vacant post, relieving another communications volunteer, or assisting in another fashion.

%%%%%%%%%%%%%%%%%%%%%%%%%%%%%%%%%%%%%%%%%%%%%%%%%%%%%%%%%%%%%%%%%%%%%%%%
%%%%%%%%%%%%%%%%%%%%%%%%%%%%%%%%%%%%%%%%%%%%%%%%%%%%%%%%%%%%%%%%%%%%%%%%

\section{Restricted Areas}

Communications volunteers should stay out of the Net Control rooms for the duration of the event. They should also refrain from exploring any areas not specifically designated for Special Olympics use.

%%%%%%%%%%%%%%%%%%%%%%%%%%%%%%%%%%%%%%%%%%%%%%%%%%%%%%%%%%%%%%%%%%%%%%%%
%%%%%%%%%%%%%%%%%%%%%%%%%%%%%%%%%%%%%%%%%%%%%%%%%%%%%%%%%%%%%%%%%%%%%%%%

\section{Leaving Assigned Post}

Communications volunteers should not leave their assigned posts until being checked out by Net Control.  If temporary coverage is needed for a short break, this can be arranged by calling Net Control a few minutes in advance.

%%%%%%%%%%%%%%%%%%%%%%%%%%%%%%%%%%%%%%%%%%%%%%%%%%%%%%%%%%%%%%%%%%%%%%%%
%%%%%%%%%%%%%%%%%%%%%%%%%%%%%%%%%%%%%%%%%%%%%%%%%%%%%%%%%%%%%%%%%%%%%%%%
%%%%%%%%%%%%%%%%%%%%%%%%%%%%%%%%%%%%%%%%%%%%%%%%%%%%%%%%%%%%%%%%%%%%%%%%
%%%%%%%%%%%%%%%%%%%%%%%%%%%%%%%%%%%%%%%%%%%%%%%%%%%%%%%%%%%%%%%%%%%%%%%%
%%%%%%%%%%%%%%%%%%%%%%%%%%%%%%%%%%%%%%%%%%%%%%%%%%%%%%%%%%%%%%%%%%%%%%%%
%%%%%%%%%%%%%%%%%%%%%%%%%%%%%%%%%%%%%%%%%%%%%%%%%%%%%%%%%%%%%%%%%%%%%%%%

\chapter{Net Control Guidelines}

\section{Start-of-Day Activities}

Prior to the start of the event, usually on the Thursday afternoon before, a team will set up the radios, antennas, feedlines, and certain other equipment in the Net Control offices.  Each day prior to starting operations, each Net Control operating point (that is, Primary, Logistics, and Backup Net Control), should perform the following checks to ensure operational readiness:

\begin{enumerate}
	\item Connect all feedlines, microphones, and other radio components.
	\item Ensure all required frequencies/channels are available for use (programmed).
	\item Perform a communications test to verify signal and audio quality on all frequencies/channels.
	\item Check that printed copies of reference materials are available, to include the Event Communications Plan, participant lists, coach contact information, event schedules, etc. %%NEED TO CLARIFY WITH OFFICIAL DOCUMENT TITLES%%
	\item Gather pens/pencils, scratch paper, and other supplies.
	\item Boot up computer(s)/tablet(s) and test wireless connectivity.  A network user ID and password will be supplied by the University of Richmond.
	\item Determine which Special Olympics Event Staff member is authorized to dial 911 in the event of an emergency.
	\item Check with the Communications Director to confirm event schedules and the Communications team staffing plan.
\end{enumerate}

If any failures or other needs are identified during this process, the Communications Director should be notified immediately.

\iffalse %%COMMENT%%
Equipment checks, opening script, etc
...and is this the best placement for this section?
\fi

%%%%%%%%%%%%%%%%%%%%%%%%%%%%%%%%%%%%%%%%%%%%%%%%%%%%%%%%%%%%%%%%%%%%%%%%
%%%%%%%%%%%%%%%%%%%%%%%%%%%%%%%%%%%%%%%%%%%%%%%%%%%%%%%%%%%%%%%%%%%%%%%%

\section{Net Control Equipment}

Equipment will be color coded for ease of installation.

Each Control station will be setup with antennas installed and ready for deployment, a 12 volt 25 or 30 amp switching power supply connected to an Anderson Powerpole expansion board and then connected to each radio. Each radio is connected via a CAT5 cable to a push button selector switch, which is then connected to a single desktop or handheld microphone.

%%%%%%%%%%%%%%%%%%%%%%%%%%%%%%%%%%%%%%%%%%%%%%%%%%%%%%%%%%%%%%%%%%%%%%%%
%%%%%%%%%%%%%%%%%%%%%%%%%%%%%%%%%%%%%%%%%%%%%%%%%%%%%%%%%%%%%%%%%%%%%%%%

\section{Opening the Net}

On the morning of the event, the Net Control Operators will get situated and prepare for the day. When ready, and at the time specified by the communications plan, the Primary Net Control Operator will open the net on the primary frequency.

\begin{quote}
	This is the Special Olympics Amateur Radio Communications Net Control.  This repeater will be in use today for the Virginia State Special Olympic Games.

	We want to thank Richmond Amateur Radio Telecommunications for the use of this repeater and their dedicated support to this venue.

	At this time we would like to take a moment of silence to remember our silent keys and athletes that have gone on before us.  (Silence) We want to thank the Amateur Operators for their spirit of volunteerism.  Net Control, Special Olympics (Call Sign)
\end{quote}

%%%%%%%%%%%%%%%%%%%%%%%%%%%%%%%%%%%%%%%%%%%%%%%%%%%%%%%%%%%%%%%%%%%%%%%%
%%%%%%%%%%%%%%%%%%%%%%%%%%%%%%%%%%%%%%%%%%%%%%%%%%%%%%%%%%%%%%%%%%%%%%%%

\section{Roll Call and Initial Check-In}

After the net has been opened, the Primary Net Control Operator should do an initial call for all stations at their post. After this initial check-in, an hourly roll call should be performed along with the periodic status reports.

%%%%%%%%%%%%%%%%%%%%%%%%%%%%%%%%%%%%%%%%%%%%%%%%%%%%%%%%%%%%%%%%%%%%%%%%
%%%%%%%%%%%%%%%%%%%%%%%%%%%%%%%%%%%%%%%%%%%%%%%%%%%%%%%%%%%%%%%%%%%%%%%%

\section{Periodic Status Reports}

Net Control will take an hourly status report from each Amateur Radio operator. The status reports will include:

\begin{itemize}
	\item Is the event on time?
	\item Is the event ahead of schedule? If so, how long?
	\item Is the event behind schedule? If so, how long?
	\item What is the estimated finish time?
	\item Supply check?
\end{itemize}

%%%%%%%%%%%%%%%%%%%%%%%%%%%%%%%%%%%%%%%%%%%%%%%%%%%%%%%%%%%%%%%%%%%%%%%%
%%%%%%%%%%%%%%%%%%%%%%%%%%%%%%%%%%%%%%%%%%%%%%%%%%%%%%%%%%%%%%%%%%%%%%%%

\section{Handling Resource Requests}

The Logistics Control operator may receive a resource request from time to time. When a request comes in, it is important to be sure to gather:

\begin{enumerate}
	\item The name of the individual (including volunteer title)
	\item The requested item(s)
	\item The number of item(s) requested
	\item The location of the items to be delivered
	\item The estimated time of the resource need.
\end{enumerate}

The Logistics Net Control Operator will contact the appropriate team member, relaying the needed information. They will then gather a response, and respond to the Amateur operator requesting the need.

%%%%%%%%%%%%%%%%%%%%%%%%%%%%%%%%%%%%%%%%%%%%%%%%%%%%%%%%%%%%%%%%%%%%%%%%
%%%%%%%%%%%%%%%%%%%%%%%%%%%%%%%%%%%%%%%%%%%%%%%%%%%%%%%%%%%%%%%%%%%%%%%%

\section{Traffic for Other Net}

Stations may inadvertently direct traffic to the wrong Net Control (for example, calling Logistics Net Control instead of the Primary Net Control).  In this situation, the answering Net Control may use discretion and either redirect the call to the correct Net Control, or handle the traffic and advise the calling station of the best route for similar calls in the future.

%%%%%%%%%%%%%%%%%%%%%%%%%%%%%%%%%%%%%%%%%%%%%%%%%%%%%%%%%%%%%%%%%%%%%%%%
%%%%%%%%%%%%%%%%%%%%%%%%%%%%%%%%%%%%%%%%%%%%%%%%%%%%%%%%%%%%%%%%%%%%%%%%

\section{Handling Medical Issues and Emergency Traffic}

Emergency traffic will usually come into Net Control in the form of a medical issue.  Logistics Net Control is responsible for handling these calls and dispatching the medical team.  The medical team can be reached on their radios.  \textbf{Upon receipt of a call for medical assistance, all communications on the repeater must be temporarily suspended until all of the details are obtained.}  Logistics Net Control must gather:

\begin{itemize}
	\item The nature of the medical issue.
	\item The exact location where medical assistance is required.
	\item The bib number of the athlete(s) in need of medical assistance.
\end{itemize}

This information must be immediately relayed to the medical team.  The repeater must be opened back up for routine traffic as quickly as possible, however \textbf{the repeater should not be released until the medical team confirms they have all of the information needed to respond to the call}.

Any calls to 911 must be completed by Special Olympics Event Staff.

\rvahbox{Calls to 911}{All three Net Control positions must be aware of the Special Olympics Event Staff member designated to call 911 in the event of an emergency.  Under no circumstances shall a member of the Communications Team place a call to 911.}

\iffalse %%COMMENT%%
We want to change emergencies.. No longer will we hold traffic for nets. We need
to get back to the "Break Break" for emergency traffic, which then we hold only
for the traffic, not until the ER is rectified. We need to work  HARD on getting
Medical onto their own freq. and allowing logistics to handle the ER Dispatch,
etc. We will be having a meeting in January with Leisha and Dave Polaski in
regards. Wold like you to be there also. More to come here as we get closer, but
Were taking control and selling the idea direct to Dave to get his buy in and
push on Dennis. no more Dennis pushing us around. They need to do their job too!
\fi %%ENDCOMMENT%%

%%%%%%%%%%%%%%%%%%%%%%%%%%%%%%%%%%%%%%%%%%%%%%%%%%%%%%%%%%%%%%%%%%%%%%%%
%%%%%%%%%%%%%%%%%%%%%%%%%%%%%%%%%%%%%%%%%%%%%%%%%%%%%%%%%%%%%%%%%%%%%%%%

\section{Handling Weather Issues}

\iffalse %%COMMENT%%
 * Clearly define process or criteria in which we begin collecting or
   disseminating weather data?
\fi %%ENDCOMMENT%%

%%%%%%%%%%%%%%%%%%%%%%%%%%%%%%%%%%%%%%%%%%%%%%%%%%%%%%%%%%%%%%%%%%%%%%%%
%%%%%%%%%%%%%%%%%%%%%%%%%%%%%%%%%%%%%%%%%%%%%%%%%%%%%%%%%%%%%%%%%%%%%%%%

\section{Backup Repeater Usage}

In addition to the primary repeater used for a majority of event traffic, a backup repeater (and/or possibly a simplex frequency) will be designated as part of the Event Communications Plan.  This repeater will be monitored continuously by Backup Net Control and all communications will shift to this backup repeater in the event of interference or an outage of the primary repeater.

All Communications volunteers should pre-program the backup repeater into their radios and must understand in advance how and when to switch to the backup repeater, as an announcement likely cannot be made over the air.

%%%%%%%%%%%%%%%%%%%%%%%%%%%%%%%%%%%%%%%%%%%%%%%%%%%%%%%%%%%%%%%%%%%%%%%%
%%%%%%%%%%%%%%%%%%%%%%%%%%%%%%%%%%%%%%%%%%%%%%%%%%%%%%%%%%%%%%%%%%%%%%%%

\section{Lost Athletes and Lost Coaches}

Every year several athletes become separated from their group. Net Control will act as the location liaison for these athletes ensuring the correct personnel are notified.  Net Control will be given a bib number for a lost athlete including their name. Net Control then contacts the Backup Net Control Operator through the internal intercom system.

The Backup Control Operator will take control of the lost athlete situation and filter the delegation/coach rooster.  Once the delegation/coach has been located in the rooster, they will be contacted via cell phone.  The Backup Net Control Operator will then notify the Amateur operator that the head of delegation/coach has been notified and is responding.  Backup Control will remind the Amateur Operator/volunteer to remain with the lost athlete until they can be reunited with their party.

%%%%%%%%%%%%%%%%%%%%%%%%%%%%%%%%%%%%%%%%%%%%%%%%%%%%%%%%%%%%%%%%%%%%%%%%
%%%%%%%%%%%%%%%%%%%%%%%%%%%%%%%%%%%%%%%%%%%%%%%%%%%%%%%%%%%%%%%%%%%%%%%%

\section{Breaks and Meal Relief}

Net Control understands the need for a short break and will make every effort to find a temporary stand-in to allow for this.  Net control, once notified by the operator of the need for a break or relief operator will do everything in their power to find a suitable replacement operator in a timely manner.  Once a replacement has been identified, that operator will be assigned to that location and Net Control will notify the current operator of their replacement.  Once the replacement is onsite, the current operator must verify release from post with Net Control.

Net Control will make every effort to ensure all operators receive their meals, calling out prior to each, and reminding operators in the field that they must request their meals from the volunteer coordinators.

Net Control will take a tally of operators for evening meals, which can then be picked up and eaten in the South Concourse, Third Floor location or the operator will be sent to eat at the dinning hall, entirely left up to the operator.

The Director will be resposible for getting meals to the Net Control Operators.

%%%%%%%%%%%%%%%%%%%%%%%%%%%%%%%%%%%%%%%%%%%%%%%%%%%%%%%%%%%%%%%%%%%%%%%%
%%%%%%%%%%%%%%%%%%%%%%%%%%%%%%%%%%%%%%%%%%%%%%%%%%%%%%%%%%%%%%%%%%%%%%%%

\section{End-of-Day Activities}

At the end of each day, the Net Control operators are responsible for ensuring the security of all equipment and that the Logistics, Security, and Medical repeaters are secure, online and functional.

A release of frequency for the local Amateur Radio repeater, returning it to normal Amateur use should be broadcast on all Amateur Radio repeaters being utilized for the event. Please thank the Amateurs for their volunteering as well as the repeater owners for the use of their systems.

A final reminder via the repeaters to Logistics, Security and Medical alike to charge batteries that evening should be completed. At that time, Net Control can notify them of our leave of station until the next morning.

The Net Control stations should be tidied up and all trash removed from the rooms.

%%%%%%%%%%%%%%%%%%%%%%%%%%%%%%%%%%%%%%%%%%%%%%%%%%%%%%%%%%%%%%%%%%%%%%%%
%%%%%%%%%%%%%%%%%%%%%%%%%%%%%%%%%%%%%%%%%%%%%%%%%%%%%%%%%%%%%%%%%%%%%%%%
%%%%%%%%%%%%%%%%%%%%%%%%%%%%%%%%%%%%%%%%%%%%%%%%%%%%%%%%%%%%%%%%%%%%%%%%
%%%%%%%%%%%%%%%%%%%%%%%%%%%%%%%%%%%%%%%%%%%%%%%%%%%%%%%%%%%%%%%%%%%%%%%%
%%%%%%%%%%%%%%%%%%%%%%%%%%%%%%%%%%%%%%%%%%%%%%%%%%%%%%%%%%%%%%%%%%%%%%%%
%%%%%%%%%%%%%%%%%%%%%%%%%%%%%%%%%%%%%%%%%%%%%%%%%%%%%%%%%%%%%%%%%%%%%%%%

\chapter{Frequently Asked Questions}

\iffalse %%COMMENT%%
 * Need to develop a FAQ - Net control runs event,
   Don't know? Ask Net control, etc... Get food, and stay hydrated! 3rd
   party traffic is allowed - know when to hand the radio to that person!
\fi %%ENDCOMMENT%%

\end{document}
